\chapter{Einleitung}
In der Vorlesung „Embedded Systeme in der Automation“ geht es darum zu lernen, wie ein Mikrocontroller objektorientiert, effektiv und mit möglichst wenig Speicherauslastung programmiert werden kann. Dazu wurden in der Vorlesung einige nützliche Methoden übermittelt. Gegen Ende des Semesters wird in diesem Modul jedoch keine Klausur geschrieben. Stattdessen bekommen die Studierenden ein Projekt zur praktischen Anwendung sämtlicher Methoden. Das Ziel dieser Projektarbeit ist es eine Ampelsteuerung zu implementieren.

\section{Aufgabenstellung}
Die in diesem Projekt zu erstellende Ampel soll, wie auch die Ampeln im Straßenverkehr, zwei Betriebszustände haben. Diese sind „active“ und „flashing“. Ersterer Zustand soll eine normale Ampelsteuerung ermöglichen mit den Farbwechseln Rot $\rightarrow$ Rotgelb $\rightarrow$ Grün $\rightarrow$ Gelb $\rightarrow$ Rot. Der zweite Zustand soll signalisieren, dass die Ampel nicht in Betrieb ist. Dies soll dadurch erreicht werden, dass das gelbe Licht blinken soll. Weiterhin soll es für den Benutzer möglich sein die Schnittstellen zur Ampelsteuerung selbst zu wählen. Hier soll es zwei Möglichkeiten geben. Erstere soll die Hardwareschnittstelle mit zwei Buttons und drei LEDs mit den Farben rot, gelb und grün sein. Andernfalls soll es auch möglich sein die Ampel über das Terminal zu steuern. Dies wäre die Softwareschnittstelle.\\

\section{Vorgehensweise}

Die Vorgehensweise bei diesem Projekt orientiert sich an einem Leitfaden, welcher von der Dozentin Claudia Heß stammt. Zunächst werden die bereitgestellte Hardware analysiert und erste Vereinbarungen bezüglich der Pinbelegungen getroffen. Danach wird ein Styleguide angelegt. Dieser ist nötig um einen möglichst einheitlichen Code zu erhalten. Im nächsten Schritt wird dann eine Architektur entworfen um alle geforderten Funktionen zu erfüllen. Daraufhin werden verwendete State Patterns vorgestellt und erklärt. Zuletzt wird dann noch auf den Hardwarezugriff eingegangen, bevor gezeigt wird, wie der Wechsel zwischen Hardware- und Softwareschnittstelle erfolgt. Zum Schluss wird dann ein Fazit gezogen.

\chapter{Arbeitsaufteilung}

Diese Arbeitsaufteilung dient als grobe Orientierung der Beschäftigung unter den Teammitgliedern und ist in nachstehender \autoref{tab:Arbeitsaufteilung} dargestellt.\

\begin{table}[H] 
	\centering
	\begin{tabular}[H]{c|c} \label{tab2}
		Teammitglied & Aufgaben \\
		\hline
		Cedric Franke & Zustandsautomat, Singleton\\
		Niklas Stein &  Wechsel zwischen Hardware- und Softwarebetrieb \\
		Timo Kempf & Hardwarezugriff  \\
		Sven Mößner & Klassendiagramm, Context, Interface und Concrete Klassen \\
	\end{tabular}
	\caption{Arbeitsaufteilung}
	\label{tab:Arbeitsaufteilung}
\end{table}